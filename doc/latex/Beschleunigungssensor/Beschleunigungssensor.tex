\subsection[Beschleunigungssensor]{Beschleunigungssensor\footnote{Jens Helge Micke}}
\thispagestyle{fancy}
\label{Beschleunigungssensor}
Nach Anwendungsfall 1.4.2 (siehe Seite \pageref{Anwendungsfälle} ff) galt es Blättern durch das Auswerten in den Endgeräten verbauter Beschleunigungssensoren zu ermöglichen.

\subsubsection{Anforderungen}
\label{BeschleunigungssensorAnforderungen}
Zur Auswertung des Beschleunigungssensors müssen dessen Daten ausgelesen werden können.\\
Relativ zur momentanen Endgerätsorientierung soll das Kippen nach Rechts oder Links erkannt und als Vor- oder Zurück-Blättern interpretiert werden.\\
Kleinstbewegungen gilt es zu ignorieren.\\
Mehrfachauslösungen sind zu minimieren.

\subsubsection{Umsetzung}
\label{BeschleunigungssensorUmsetzung}
An dieser Stelle werden kurz die Ansätze der Beschleunigungssensorsteuerung skizziert.\\
Die Implementierung ist im Quellcode app\textunderscore gui/pages/PdfSteuerung.qml nachzulesen.
\paragraph{Auslesen des Beschleunigungssensors}$\;$\\
Nach Integration von QtSensors kann der Sensortyp Accelerator genutzt und eingestellt werden.\\
Unter den Einstellungen können der Sensor aktiviert, die Datenrate\footnote{Nicht von allen Beschleunigungssensoren unterstützt.} eingestellt und das Verhalten bei einem Sensorveränderungsereignis definiert werden.\\
Die Sensorwerte Der X-, Y- und Z-Achse sind über .reading.x .reading.y .reading.z auszulesen.
\paragraph{Endgerätsorientierung}$\;$\\
Zur Erkennung der derzeitigen Endgerätsorientierung stellt Qt QtQuick.Window bereit.
Durch die Definition einer Updatemaske können die Wechsel zwischen den Orientierungen 1: Portrait, 2: Landscape, 4: Inverted Portrait und 8: Inverted Landscape erkannt und über Screen.Orientation ausgelesen werden.
\paragraph{Kleinstbewegungen}$\;$\\
Kleinstbewegungen werden am einfachsten über eine definierte Mindestauslenkung des Beschleunigungssensors ignoriert.
\paragraph{Mehrfachauslösung}$\;$\\
Um Mehrfachauslösungen vorzubeugen und die Beschränkung der nicht immer einstellbaren Datenrate zu umgehen kann mit QtQuick ein Zeitgeber definiert werden der das Blättern nur alle 500ms zulässt.
\subsubsection{Alternativen}
\label{BeschleunigungssensorAlternativen}
Technisch aufwendigere Alternativen sind denkbar, für den Umfang der Projektarbeit und mit Blick auf die Rechenleistung der Endgeräte jedoch nicht unbedingt Zielführend.\\
An dieser Stelle seien einige Möglichkeiten motiviert.
\paragraph{Kleinstbewegungen}$\;$\\
Kleinstbewegungen und die Eigenschwingung des Endgerätes könnten auch über eine Spektralanalyse des Sensorsignals ignoriert werden.
\paragraph{Mehrfachauslösung}$\;$\\
Der Besprochene Mehrfachauslöseschutz könnte auch Global eingesetzt werden.