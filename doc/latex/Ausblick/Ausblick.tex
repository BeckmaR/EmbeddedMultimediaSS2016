\chapter[Zusammenfassung und Ausblick]{Zusammenfassung und Ausblick\footnote{Jens Helge Micke}}
\thispagestyle{fancy}
\label{Ausblick}
Das Projekt verlief mit seinen Höhen und Tiefen erfolgreich.\\
Am Ende Stand ein funktionstüchtiges Produkt.\\
\section{Gelerntes}
Bei der Bearbeitung des Projektes wurden viele neue Erfahrungen gesammelt.\\
\begin{itemize}
	\item Entwicklung eingebetteter System
	\item Projektierung von Anfang bis Ende
	\item Erste Begegnung mit QT
	\item Entwickeln für Android, GNU/Linux, Windows
	\item Arbeiten mit Versionskontrollsoftware
	\item Arbeit im Team
	\item Kennenlernen von Server/Client Strukturen
	\item Angewandte Signalverarbeitung
	\item Kreative Problemlösungen
	\item Umgang mit Frustration und Erleichterung
\end{itemize}
\section{Weitere Möglichkeiten}
Während der letzten Phasen des Projektes wurden weitere Einsatzmöglichkeiten des Produktes offensichtlich.
\paragraph{Digital Signage/Marquee}$\;$\\
Mit der kleinen Erweiterung um einen automatisch Blätternden Klienten kann das Projekt als Digitale Werbefläche umfunktioniert werden.
\paragraph{Medienverbreiter und HotSpot}$\;$\\
In seiner jetzigen Form könnte das Produkt auch als HotSpot und Menükarte oder Verbreitungsplatform für Bücher im Gastronomiebereich dienen.\\
Mit Erweiterung des Servers auch direkt als Bestellplatform.
\paragraph{DeadDrop, GeoCache}$\;$\\
Solarbetrieben in der Wildnis oder auf privaten Grund kann das Produkt auch als digitale Schatztruhe für kleine und große Abenteurer dienen.
\section{Schlusswort}
Gruppe 5 dankt Herrn Dr. Wolfgang Theimer für diese Möglichkeit Praxiserfahrungen zu sammeln.